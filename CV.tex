% !TEX program = xelatex
\documentclass{muratcan_cv}
\usepackage{enumitem}
\usepackage{natbib}
\usepackage{bibentry}
\usepackage{hyperref}
\bibliographystyle{unsrt}

\setname{}{Josh W. Day}
\setaddress{Altadena, CA}
\setmobile{(916) 765-9396}
\setmail{jwday.work@gmail.com}
\setposition{Work Student} %ignored for now
\setlinkedinaccount{https://www.linkedin.com/in/joshwday} %you can play with color of the template (red is also nice..)
\setgithubaccount{https://github.com/jwday} %you can play with color of the template (red is also nice..)
\setthemecolor{Orange} %you can play with color of the template (red is also nice..)
\urlstyle{same}

%----------------------------------------------------------------------------------
%            bibliographyish
%----------------------------------------------------------------------------------
\begin{filecontents}{publications.bib}
	@article{day2020,
		author={Day, Josh W and Robinson, Stephen K},
		title={{Two-Fault} {Tolerant} {Cold} {Gas} {Propulsion} {System} for {Spacecraft-Inspection} {CubeSat}},
		booktitle={AIAA Scitech 2020 Forum},
		pages={1665},
		year={2020},
		note={DOI: \url{https://doi.org/10.2514/6.2020-1665}}
  	}  
\end{filecontents} 


\begin{document}
%Set variables
%You can add sections, texts, explanations just by copying the style below. Replace the dummy texts "\lipsum[1][x-x]\par" with actual texts.
%Create header
\headerview
\vspace{1em}
%Sections
%
% Summary
% \addblocktext{Summary}{%
% \lipsum[1][1-12]\ %replace this part with actual text
% }
%
% Education
\section{Education}
    \datedexperience{Master of Science}{University of California, Davis -- Sep 2020}
	\explanation{Mechanical and Aerospace Engineering}
	\explanationdetail{%
		\begin{itemize}[label=\textcolor{red}{$\circ$}, leftmargin=*]
			\item Thesis: \emph{Development of a Two-Fault Tolerant Cold Gas Propulsion System and Air Bearing\newline Testbed for Application to a Spacecraft-Inspection CubeSat}, \href{https://github.com/jwday/Thesis-Pub/raw/master/Josh\%20Day\%2C\%20MS\%20Thesis\%20vFinal.pdf}{available online.}
		\end{itemize}
		\vspace{-1.3em}
	}
    \datedexperience{Bachelor of Science (Double Major)}{University of California, Davis -- Jun 2013}
    \explanation{Mechanical Engineering, Aerospace Science \& Engineering}
    % \explanation{}



% Experience
\section{Experience}
	\datedexperience{NASA Jet Propulsion Laboratory}{May 2021 -- Present}
	\explanationwithlocation{Mechatronics Engineer}{Pasadena, CA}
	\explanationsubsection{Orbiting Sample Container Lid Restraint And Release - Lead Mechanism Designer}
	\explanationdetail{%
	\begin{itemize}[label=\textcolor{red}{$\circ$}, leftmargin=*]
			\setlength\itemsep{-0.05em}
			% \item Led a 3-person team to implement a launch restraint mechanism and steered the design through PDR.
			\item Developed MATLAB- and Python-based tools to analyze contact stress and joint friction, simulate mechanism dynamics during deployment, and ensure positive force and torque margins with respect to NASA specs and safety factors.
			\item Modeled and simulated the dynamic motion of the release of a launch restraint to characterize and design to avoid race conditions in deployment mechanisms.
		\end{itemize}
	}
	\explanationsubsection{Orbiting Sample Container - Support Engineer}
	\explanationdetail{%
		\begin{itemize}[label=\textcolor{red}{$\circ$}, leftmargin=*]
			\item Guided prototype latch mechanism through manufacturing and Instron testing by creating drawings, selecting materials, and coordinating test campaign leading to successful validation of capabilities up to 10 kN.
			\item Determined sample count, designed support hardware, created test matrix, supervised testing, and processed test data for 12 unique crushable materials used for energy absorption to maintain sample integrity.
		\end{itemize}
	}
	\explanationsubsection{Thermal Control Integrated Pump Assembly - Support Engineer}
	\vspace{0.5em}
	\explanationdetail{%
		\begin{itemize}[label=\textcolor{red}{$\circ$}, leftmargin=*]
			\setlength\itemsep{-0.05em}
			\item Delivered flight hardware components by overseeing receiving inspection and vendor processes, and managed inventory database to track progress and schedule.
			\item Analyze time-history data from pump start/stop testing by looking at changes in current, pressure, and flow rate over the lifetime, and reported findings directly to Product Delivery Manager.
			\item Developed MATLAB tool to batch-process magnetic runout test data across twleve trials for each component, which required the data to be automatically resampled, filtered, and averaged together.
			\item Developed procedures for and performed requirements validation testing for 
			% \item Developed MATLAB- and Python-based tools to analyze contact stress, joint friction, and simulate mechanism dynamics to study race conditions in deployment processes.
			% \item Designed and constructed prototype securement mechanism for Mars Sample Return (MSR) orbiting sample (OS) canister, and coordinated test campaign to characterize performance which led to immediate design refinements.
			% \item Designed and performed static and modal analyses using NX NASTRAN on OS mass model which will undergo repeated high velocity impacts, and made responsible for interface between mass model and impactor hardware.
			% \item Developed, coordinated, and managed test campaign to characterize stress-strain response of crushable materials and programmed Python-based data processing scripts as part of OS design trade study.
			% \item Documented and routinely presented design reviews, test readiness reports, test results senior engineers
			% \item Supported fabrication and delivery of pumps for the Europa Clipper Heat Redistribution System (HRS) by tracking components in the assembly lifecycle, writing MATLAB-based tools to assess magnetic runout on pump rotors, and coordinating with outside vendors to supply parts.
			% \item You can (and should) add more here . You're in aerospace, you're always trying to save mass. Instead talk about "what were the capabilities" and describe how you got the design through PDR or something
			% \item The prototype isn\'t quantifiable, it just IS. But the requirements that guided it ARE quantifiable and you designed to it.
			% \item Designed and tested a prototype latch mechanism for the Mars Sample Return orbiting sample canister
			% \item Orchestrated test campaign for crushable materials, aiming to find the best material without a specific baseline comparison.
			% \item Designed test bed to fasciliate optimization of energy absorber
			% \item Designed a test campaign for 12 unique crushable materials to optimize the design for mass efficiency and energy absorption capability while maintaining Mars soil sample integrity.
			% \item Efficiently managed paint coupon durability and impact resistance tests, evaluating a single sample for each coupon type.
			% \item Oversaw vendor manufacturing processes, performed requirements validation testing, and oversaw receipt inspection of components as cognizant engineer for within the Europa Clipper Heat Redistribution System.
			% \item Delivered flight hardware components for the Europa Clipper Heat Redistribution System by overseeing receiving inspection and vendor processes, and writing procedures for and performing requirements validation testing.
		\end{itemize}
	}
	%
	\datedexperience{Human/Robotics/Vehicle Integration \& Performance Lab}{Jun 2016 -- Sep 2020}
	\explanationwithlocation{Graduate Student Researcher}{Davis, CA}
	\explanationdetail{%
		\begin{itemize}[label=\textcolor{red}{$\circ$}, leftmargin=*]
			\setlength\itemsep{-0.05em}
			\item Defined requirements for and designed, simulated, and constructed a small satellite propulsion system prototype
			\item Utilized something-something equations to define mass and volume requirements for ... . Designed, simulated, and constructed ...
			\item Custom-built a 3 degree-of-freedom air bearing platform to simulate microgravity for propulsion system testing using COTS hardware and 3D-printed components.
			\item Architected and integrated propulsion system control hardware and developed full stack human-machine interface for real-time wireless control and feedback using Arduino and Raspberry Pi.
			\item Implemented Python + OpenCV-based computer vision tracking system to measure air bearing platform motion.
			\item Utilized Kalman filtering to estimate acceleration, enabling low-cost vision-based thrust measurement.
			% \item Developed Python-based gas dynamics simulation to predict time-varying performance of thrusters with non-constant supply pressure, experimentally validated the predictions, and presented the results at AIAA SciTech 2020.
			% \item Designed and constructed a cold gas propulsion system using 3D printed pressure vessels and commercial off-the-shelf parts to achieve full 3 degrees-of-freedom control of an air bearing test platform.
		\end{itemize}
		% \vspace{-1.3em}
	}
	%
	\explanationwithlocation{Conference Publication}{AIAA SciTech 2020, Orlando, FL}
	\explanationdetail{%
		\begin{itemize}[label=\textcolor{orange}{$\circ$}, leftmargin=*]
			\setlength\itemsep{-0.05em}
			\vspace{-0.9em}
			\item \nobibliography{publications} \bibentry{day2020}
		\end{itemize}
	}
	%
	\explanationwithlocation{Teaching Assistant}{Sep 2015 -- Mar 2019 (10 qtrs.)}
	\vspace{0.5em}
	\explanationdetail{%
		\begin{itemize}[label=\textcolor{orange}{$\circ$}, leftmargin=*]
			\setlength\itemsep{-0.05em}
			\vspace{-0.9em}
			\item \emph{Engineering Design \& Communications} (Lead) \quad \textcolor{red}{$\circ$} \emph{Measurement Systems} \quad \textcolor{red}{$\circ$} \emph{Manufacturing Processes}
		\end{itemize}
	}
	% \explanationwithlocation{Patent Pending}{April 2021}
	% \explanationdetail{%
	% 	\begin{itemize}[label=\textcolor{orange}{$\circ$}, leftmargin=*]
	% 		\setlength\itemsep{-0.05em}
	% 		\vspace{-0.9em}
	% 		\item A 3D-printed nested-plenum design for a two-fault tolerant cold gas propulsion system.
	% 	\end{itemize}
	% }
	% %
    % \datedexperience{UC Davis Department of Mechanical and Aerospace Engineering}{Sep 2015 -- Jun 2019}
    % \explanationwithlocation{Teaching Assistant (Multiple Courses)}{Davis, CA}
	% \explanationdetail{%
	% 	\begin{itemize}[label=\textcolor{red}{$\circ$}, leftmargin=*]
	% 		\setlength\itemsep{-0.05em}
	% 		% \item As lead TA for Engineering Design \& Communications, assisted with lesson plan development and supervised three bi-weekly lab sessions consisting of 6 teams of 4 students, each with their own individual project needs.
	% 		% \item Applied experience with programming, sensors, and actuators to introduce students to data collection with microcontrollers and guide team projects to address problems identified on the UC Davis Student Farm.
	% 		% \item Provided effective feedback to students engaged in public speaking and presentations, citing notable improvement in technical communication and public speaking skills from beginning to end of quarter.
	% 		\item \emph{Engineering Design \& Communications} -- As lead TA, assisted in lesson plan development, gave weekly lectures introducing sensors and microcontrollers, and provided critical feedback on public speaking and technical communication.
	% 		\item \emph{Measurement Systems} -- Supervised weekly lab sessions introducing students to electro-mechanical sensors such as op-amps and strain gauges to measure bending stress and accelerometers to identify vibrational modes.
	% 		\item \emph{Manufacturing Processes} -- Instructed students on safe operation of manual lathing and milling, utilizing geometric design and tolerancing (GD\&T), and developing process plans to fabricate steel and aluminum gyroscopes.
	% 	\end{itemize}
	% }
    %
    \datedexperience{NASA Jet Propulsion Laboratory}{Jan 2017 -- Aug 2017, Jun 2018 -- Aug 2018}
	\explanationwithlocation{Mechanical Engineer Co-op (2017), Intern (2018)}{Pasadena, CA}
	\vspace{0.5em}
	\explanationdetail{%
		\begin{itemize}[label=\textcolor{red}{$\circ$}, leftmargin=*]
			\setlength\itemsep{-0.05em}
			% \item Assisted with performance testing and evaluation of flight-like motors and actuators in Mars-like environments and assisted with design and execution of special testing to uncover cause of premature gearbox degradation.
			% \item Developed Python and MATLAB-based data analysis tools to visualize motor performance test data and generate test summaries to be delivered directly to supervising engineers.
			% \item Designed mass models of flight components and provided sensor recommendations for launch vibration testing of components for the Mars 2020 rover Sample Caching Subsystem.
			% \item Assembled and calibrated bench-top dynamometer setup to be used for motor/actuator performance testing. Documented assembly process and provided recommendations for improvement to supervising engineer.
			% \item Developed procedure and coordinated assembly of epoxy-bonded aluminum honeycomb shock dampers, personally handling and successfully delivering the hardware which was flown on the Mars 2020 Perseverance rover.
			\item Flight hardware performance validation, data analysis, and hardware assembly for the Mars 2020 exploration rover.
		\end{itemize}
	}		
    %
    % \datedexperience{NASA Jet Propulsion Laboratory}{Jun 2018 -- Aug 2018}
	% \explanationwithlocation{Mechanical Engineering Intern}{Pasadena, CA}
	% \explanationdetail{%
	% 	\begin{itemize}[label=\textcolor{red}{$\circ$}, leftmargin=*]
	% 		\setlength\itemsep{-0.05em}
	% 		\item Designed components to emulate mass properties of flight hardware and provided sensor recommendations for launch vibration testing of components for the Mars 2020 rover Sample Caching Subsystem.
	% 		\item Assembled and calibrated a customized bench-top dynamometer to be used for motor/actuator performance testing, documenting assembly process and providing recommendations for improvement to supervising engineer.
	% 		\item Developed procedure and coordinated assembly of epoxy-bonded aluminum honeycomb shock dampers, personally handling and successfully delivering the hardware which was flown on the Mars 2020 Perseverance rover.
	% 	\end{itemize}
	% }		
    % %
    % \datedexperience{NASA Jet Propulsion Laboratory}{Jan 2017 -- Aug 2017}
	% \explanationwithlocation{Mechanical Engineering Co-op}{Pasadena, CA}
	% \explanationdetail{%
	% 	\begin{itemize}[label=\textcolor{red}{$\circ$}, leftmargin=*]
	% 		\setlength\itemsep{-0.05em}
	% 		\item Assisted with performance testing and evaluation of flight-like motors and actuators in Mars-like environments, and assisted with design and execution of special testing to uncover cause of premature gearbox degradation.
	% 		% \item Drew schematics for and fabricated custom power and data wiring harnesses to connect motion controllers and sensors for Mars 2020 motor and actuator testing.
	% 		\item Developed Python and MATLAB-based data analysis tools to visualize motor performance test data and generate test summaries to be delivered directly to supervising engineers.
	% 		\item Supported R\&D on the proposed Mars Sample Return mission with trade studies and conceptual design development for a robust latching system and alignment/release mechanism.
	% 	\end{itemize}
	% }
	% %
    \datedexperience{SP Engineering Services}{May 2014 -- Aug 2015}
	\explanationwithlocation{Product Design \& Fabrication Engineer}{Vacaville, CA}
	\explanationdetail{%
		\begin{itemize}[label=\textcolor{red}{$\circ$}, leftmargin=*]
			\setlength\itemsep{-0.05em}
			% \item Collaborated with a team of engineers to designed and fabricate FDA-compliant storage and organizational solutions for large-scale pharmaceutical manufacturing utilizing CAD/CAM software and CNC machinery.
			% \item Operated CNC router and laser cutter on daily basis and established standard operating procedures to fabricate FDA-compliant products for large-scale pharmaceutical manufacturing and biotech customers.
			% \item Performed R\&D for proprietary fabrication process employing photography techniques and image manipulation to quickly generate CAM tool paths and enabling rapid turnaround of customized products.
			% \item Interacted directly with customers to identify needs, define requirements, and implement design changes based on feedback.
			% \item Collaborated with a small team of engineers to design and manufacture FDA-compliant storage, organizational, and process solutions for pharmaceutical manufacturing and biotechnology research organizations.
			% \item Developed novel process to rapidly dimensionalize client hardware by employing photography and image manipulation to quickly generate CAM tool paths, which reduced the product turnaround time from days to hours.
			% \item Lead product ideation and prototyping efforts to take advantage of available CNC machine down-time and fabricated conceptual bamboo and plastic-based products for the beverage and wine industry.
			\item Designed and fabricated FDA-compliant storage solutions, operated CNC machinery, performed R\&D for rapid product turnaround, and engaged directly with customers.
		\end{itemize}
	}
	


% Publications
% \section{Publications}
% 	\mbox{
% 		\hspace{1em}
% 		\begin{minipage}{.95\linewidth}
% 			\footnotesize
% 			% \scshape
% 			\nobibliography{publications}
% 			\bibentry{day2020}
% 		\end{minipage}
% 	}\\
	%

% Skills
\section{Technical Skills}
    % %
    % \newcommand{\skillone}{\createskill{CAD/CAM}{SolidWorks \cpshalf NX 11 \cpshalf Slic3r \cpshalf BobCAM \cpshalf ESPRIT}}
    % %
    % \newcommand{\skilltwo}{\createskill{Manufacturing}{Lathing \cpshalf Milling \cpshalf 3D Printing \cpshalf Printed Circuit Boards}}
    % %
    % \newcommand{\skillthree}{\createskill{Programming}{Python \cpshalf C++/Arduino \cpshalf MATLAB \cpshalf JavaScript/HTML/CSS}}
    % %
    % \newcommand{\skillfour}{\createskill{Development}{Git \cpshalf Linux/WSL \cpshalf VS Code}}
    % %
	% \createskills{\skillone, \skilltwo, \skillthree, \skillfour}
	\explanationdetail{\footnotesize \textsc{%
		\begin{tabular}{ r l l }
			% \textbf{Manufacturing Experience}		& Printed Circuit Board Design \cpshalf Wiring Harness Fabrication \cpshalf Soldering \cpshalf 3D Printing \cpshalf CNC		\\
			% \textbf{Programming Languages}			& Python \cpshalf C++ (Arduino) \cpshalf MATLAB \cpshalf JavaScript \cpshalf HTML \cpshalf CSS 		\\
			% \textbf{Development Tools}				& Git \cpshalf Linux / WSL \cpshalf VS Code															\\
			% \textbf{CAD/CAM Software}				& Autodesk EAGLE \cpshalf Microsoft Visio \cpshalf SolidWorks \cpshalf NX 11 \cpshalf Slic3r
			\textbf{Design/Modeling}				& Siemens NX \cpshalf SolidWorks \cpshalf Autodesk EAGLE \cpshalf Slic3r 	\\
			\textbf{Programming/Software}			& Python \cpshalf Arduino (C++) \cpshalf MATLAB \cpshalf Git \cpshalf Linux / WSL \cpshalf JavaScript \cpshalf HTML \cpshalf CSS			\\
			\textbf{Manufacturing Experience}		& Mechanical Assembly \cpshalf CNC Machining \cpshalf 3D Printing \cpshalf PCB Design \cpshalf Wiring Harness Fabrication \\
			\textbf{Engineering \& Communication}	& Mechanism Design \& Implementation \cpshalf Sensor Specification \& DAQ Design \\
													& GD\&T per ASME Y14.5 \cpshalf Procedure Creation \& Execution \cphshalf Experimental Design
		\end{tabular}
		}
	}

	
% Awards
% \section{Awards and Certifications}
% 	\explanationdetail{%
% 		\begin{itemize}[label=\textcolor{red}{$\circ$}, leftmargin=*]
% 			\setlength\itemsep{-0.1em}
% 			\item Tau Beta Pi Engineering Honor Society Member, California Lambda Chapter
% 			\item EIT/FE (California), Certification No. EIT 146122
% 			\item Eagle Scout, Boy Scouts of America
% 		\end{itemize}
	% }
    % \newcommand{\extraone}{%
    % \lipsum[1][7-8]\par %replace this part with actual text
    % }
    % %
    % \newcommand{\extratwo}{%
    % \lipsum[1][9-10]\par %replace this part with actual text
    % }
    % %
    % \newcommand{\extrathree}{%
    % \lipsum[1][11-12]%replace this part with actual text
    % }
    % %
    % \newcommand{\listofextras}{\extraone, \extratwo, \extrathree}
    % %
	% \createbullets{\listofextras}
	
% \pagebreak

\section{Selected Projects}
	% \mbox{
	% 	\hspace{1em}
	% 	\begin{minipage}{.96\linewidth}
	% 		\explanationdetail{%
	% 			The following is a selection of relevent engineering projects and a brief description for each.
	% 			A link to the corresponding GitHub repository is available if viewing this document on a computer and wish to view the details, source code, and other materials for each project.
	% 		}
	% 	\end{minipage}
	% }
	% %
	% \linebreak
	% %
	\datedexperience{\href{https://github.com/jwday/ABTestbed}{\color{black}{Air Bearing Propulsion System Testbed \faGithubSquare}}}{}
	\explanationdetail{%
		Developed an air bearing-based platform and table to provide a free-floating low friction environment for testing a prototype cold gas propulsion system.
		Pneumatic components were acquired off-the-shelf and structural components 3D printed to reduce cost and facilitate dissemination of open-sourced part files.
		The air bearing surface was constructed by placing a sheet of regular window pane glass atop an optics table and shimmed as needed to achieve a functionally level surface.
		% The platform was found to operate marginally at 30 psig with improved functionality up to 50 psig, after which no performance improvements were observed.
		It was found that debris and inconsistencies in the table surface were the main contributors to degraded performance, and new facilities are being constructed to house the table and mitigate surface debris.
		% The system was required to operate independent of any external power or gas supplies and maintain a small footprint to maximize usable table space.
		% The design of the platform was driven by performance and cost constraints of the air bearings
		% Three air bearings were mounted to threaded ball studs in a triangular configuration and held in place by a 3D-printed structure.
		% This same structure supported 2x 9-oz. CO$_{2}$ bottles, a regulator, an air filter, and the propulsion system.
		% The air bearing table was constructed by placing a 0.125''-thick pane of glass atop a 3' x 5' optics bench and shimmed with 0.1-mm Kapton tape to minimize surface deviations.
		}
		%
		\linebreak
		%
		\datedexperience{\href{https://github.com/jwday/remoteProp}{\color{black}{Arduino and Raspberry Pi-based Wireless Propulsion Controller \faGithubSquare}}}{}
		\explanationdetail{%
		Constructed a wireless electronic control system for the cold gas propulsion system prototype using a NodeMCU microcontroller to drive valves via motor controllers and collect data from pressure transducers.
		% The circuitry was packed on a custom-designed printed circuit board (PCB).
		The NodeMCU was chosen as a low-cost platform which could wirelessly communicate with a Raspberry Pi-based message broker using an MQTT framework.
		NodeJS was used on the Raspberry Pi to simultaneously host the MQTT message broker and an HTTP server to allow users to send commands to the propulsion system via a web interface.
		The same framework was used to support data collection during static thrust testing by wirelessly relaying strain gauge and pressure measurements to the broker which then stored the data in a database for post-processing.
		}
		%
		\linebreak
		%
		\datedexperience{\href{https://github.com/jwday/ComputerVision}{\color{black}{Computer Vision System and Kalman Filter Data Post-processor \faGithubSquare}}}{}
		\explanationdetail{%
			Developed a Python-based computer vision algorithm to measure position, velocity, and acceleration of a free-floating air bearing platform from recorded video.
			The algorithm detected a fiducial marker placed on the platform and returned a timeseries dataset of relative pose for each frame in the video.
			This dataset was then passed to a Kalman filter which combined the data with a kinematic model to estimate the velocity and acceleration of the platform over time.
			These tools allowed a user to detect accelerations due to thrust and friction and quantitatively compare the effects of different combinations of thruster failures.
		}
		%
		\linebreak
		%
		\datedexperience{\href{https://github.com/jwday/nozzleDesign}{\color{black}{Gas Dynamics Simulation for Single Plenum Discharge \faGithubSquare}}}{}
		\explanationdetail{%
			Developed a Python-based gas dynamics simulation to predict the performance characteristics of the plenum-based propulsion system by modeling the change in propellant thermodynamic properties during a discharge process.
			The change in gas properties with each time step was modeled assuming an isentropic process for an ideal gas, and the corresponding thruster performance was calculated assuming an isentropic process through the nozzle.
			Further development replaced the isentropic model inside the plenum with interpolated thermodynamic data for real gasses to more accurately predict the temperature change during a discharge.
			Simulation results were experimentally validated by measuring pressure and thrust during a discharge, which were found to be in good agreement after the completion of an initial start-up phase whereby the mass flow rate and thrust increased to a maximum before decaying.
			% The modeled process was found to be in the same family as that of a rocket nozzle during its shut-off phase, with the exception of a brief period in the beginning where mass flow rate and thrust must first rise to a maximum before decaying.
		}
		%
		\linebreak
		%
		\datedexperience{\href{https://github.com/jwday/3D-printed-plenum}{\color{black}{3D Printed Propellant Pressure Vessel \faGithubSquare}}}{}
		\explanationdetail{%
			Constructed propellant storage volumes ('plenums') to store gaseous CO2 at 100 psig using an off-the-shelf benchtop 3D printer for use in on-ground propulsion system testing.
			A process was developed to finish and seal the 3D printed components with two-part epoxy and securely bond a threaded fitting to allow the plenums to be mated directly to the propulsion control valves.
			Aided by CAD-based stress analysis, several iterations were produced with varying connectors, geometry, and print methods in order to produce a final version which could be reliably printed in a short amount of time.
			Each plenum underwent proof testing to double its expected pressure (200 psig) using water to minimize explosive energy in event of failure.
		}
		%
		\linebreak
		%
		\datedexperience{\href{https://github.com/jwday/rotating-torus-sympy}{\color{black}{Dynamics Modeling of a Rotating Torus with Time-Varying Mass Distribution \faGithubSquare}}}{}
		\explanationdetail{%
			A Multibody Dynamics class project which used Python to implement Kane's Method of dynamic analysis to model a rotating torus in free space as system mass is shifted to offset the center of gravity.
			The motivation for this project was to explore the use of rotation to simulate gravity in a theoretical space station design and explore how internal disturbances could affect the dynamics of the rotation.
			A simply-rotating body with point masses around the circumference was modeled using the appropriate formulation, but the specific challenge lied in accessing the location of the point masses during the integration routine and changing them mid-step.
			% As expected, entirely in-plane motion resulted in simple translation of the center of mass while rotation remained in-plane.
			% More interestingly, some out-of-plane motion of the point masses resulted in a visible ``wobble'', necessitating the introduction of counter-balance masses which would be used to preserve system stability.
		}
		%
		\linebreak
		%
		\datedexperience{\href{https://github.com/jwday/LCARS-bot}{\color{black}{LCARS: A Discord Bot Using Discord.js API (Personal Project) \faGithubSquare}}}{}
		\explanationdetail{%
			A personal Discord bot created using the Discord.js library as an exercise to become more familiarzed with javaScript, specifically non-sequential function calls.
			Main features include chat-based Google search, playing YouTube audio, and an emoji react-controlled ``sound board'' to prompt the bot to play sound effects when a chat message is reacted to with pre-specified emoji(s).
			Specific challenges involved tightly managing event listener creation to prevent different instances from interfering with each other, using javaScript 'cavas' library to create Star Trek-inspired logos, and learning to use 'async' and 'await' functions to handle responsive delays (such as in API calls).
		}




%Footnote
% \createfootnote
\end{document}
