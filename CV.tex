% !TEX program = xelatex
\documentclass{muratcan_cv}
\usepackage{enumitem}
\usepackage{natbib}
\usepackage{bibentry}
\usepackage{hyperref}
\bibliographystyle{unsrt}

\setname{Josh W.}{Day}
\setaddress{625 G Street -- Davis, CA}
\setmobile{(916) 765-9396}
\setmail{jwday@ucdavis.edu}
\setposition{Work Student} %ignored for now
\setlinkedinaccount{https://www.linkedin.com/in/josh-day-49b10a58/} %you can play with color of the template (red is also nice..)
\setgithubaccount{https://github.com/jwday} %you can play with color of the template (red is also nice..)
\setthemecolor{Orange} %you can play with color of the template (red is also nice..)

%----------------------------------------------------------------------------------
%            bibliographyish
%----------------------------------------------------------------------------------
\begin{filecontents}{publications.bib}
	@article{day2020two,
		author={Day, Josh W and Robinson, Stephen K},
		title={{Two-Fault} {Tolerant} {Cold} {Gas} {Propulsion} {System} for {Spacecraft-Inspection} {CubeSat}},
		booktitle={AIAA Scitech 2020 Forum},
		pages={1665},
		year={2020},
		note={DOI: \url{https://doi.org/10.2514/6.2020-1665}}
  	}  
\end{filecontents} 


\begin{document}
%Set variables
%You can add sections, texts, explanations just by copying the style below. Replace the dummy texts "\lipsum[1][x-x]\par" with actual texts.
%Create header
\headerview
\vspace{1em}
%Sections
%
% Summary
% \addblocktext{Summary}{%
% \lipsum[1][1-12]\ %replace this part with actual text
% }
%
% Education
\section{Education}
    \datedexperience{M.S. Mechanical and Aerospace Engineering}{University of California, Davis -- 2020}\\[-0.4cm]
	% \explanation{University of California, Davis}
	\explanationdetail{%
		\begin{itemize}[label=\textcolor{red}{$\circ$}, leftmargin=*]
			\item Thesis: \emph{Development of a Two-Fault Tolerant Cold Gas Propulsion System and Air Bearing\newline Testbed for Application to a Spacecraft-Inspection CubeSat}, \href{https://github.com/jwday/Thesis-Pub/raw/master/Josh\%20Day\%2C\%20MS\%20Thesis\%20vFinal.pdf}{available online.}
		\end{itemize}
    }\\[-0.5cm]
    \datedexperience{B.S. Mechanical Engineering}{University of California, Davis -- 2013}\\[-0.6cm]
    \datedexperience{B.S. Aerospace Science \& Engineering}{}\\[-0.5cm]
    % \explanation{University of California, Davis}\\[-0.8cm]


% Publications
\section{Publications}
	\mbox{
		\hspace{1em}
		\begin{minipage}{.85\linewidth}
			\footnotesize
			% \scshape
			\nobibliography{publications}
			\bibentry{day2020two}
		\end{minipage}
	}\\[0.1cm]


% Experience
\section{Experience}
	\datedexperience{Human/Robotics/Vehicle Integration \& Performance Lab}{Jun 2016 -- Oct 2020}\\[-0.6cm]
	\explanation{Graduate Student Researcher}
	\explanationdetail{%
		\begin{itemize}[label=\textcolor{red}{$\circ$}, leftmargin=*]
			\setlength\itemsep{-0.1em}
			\item Constructed small satellite propulsion system testbed using air bearings and 3D printed components to simulate microgravity conditions, enabling highly repeatable motion measurements at low cost.
			\item Implemented Python + OpenCV-based computer vision tracking system to measure air bearing platform motion and utilized Kalman filtering to extract acceleration data for accurate vision-based thrust measurement.
			\item Developed full stack human-machine interface using JavaScript + Arduino for real-time wireless control of propulsion system and real-time feedback from sensors.
		\end{itemize}
	}
	%
    \datedexperience{UC Davis Department of Mechanical and Aerospace Engineering}{Sep 2015 -- Jun 2019}\\[-0.6cm]
    \explanation{Teaching Assistant (Various Courses)}
	\explanationdetail{%
		\begin{itemize}[label=\textcolor{red}{$\circ$}, leftmargin=*]
			\setlength\itemsep{-0.1em}
			% \item As lead TA for Engineering Design \& Communications, assisted with lesson plan development and supervised three bi-weekly lab sessions consisting of 6 teams of 4 students, each with their own individual project needs.
			\item \emph{Engineering Design \& Communications} -- As lead TA, assisted lesson plan development, gave weekly lectures introducing students to sensors and microcontrollers, and provided critical feedback on public speaking and technical communication.
			% \item Applied experience with programming, sensors, and actuators to introduce students to data collection with microcontrollers and guide team projects to address problems identified on the UC Davis Student Farm.
			% \item Provided effective feedback to students engaged in public speaking and presentations, citing notable improvement in technical communication and public speaking skills from beginning to end of quarter.
			\item \emph{Measurement Systems} -- Supervised weekly lab sessions introducing students to electro-mechanical sensors such as op-amps and strain gauges to measure bending stress and accelerometers to identify vibrational modes.
			\item \emph{Manufacturing Processes} -- Instructed students on safe operation of manual fabrication equipment and proper use of geometric dimensioning and tolerancing, leading each student in the construction of their own personalized gyroscope.
		\end{itemize}
	}
    %
    \datedexperience{NASA Jet Propulsion Laboratory}{Jun 2018 -- Aug 2018}\\[-0.6cm]
	\explanation{Mechanical Engineering Intern}
	\explanationdetail{%
		\begin{itemize}[label=\textcolor{red}{$\circ$}, leftmargin=*]
			\setlength\itemsep{-0.1em}
			\item Designed components to emulate mass properties of flight hardware and provided sensor recommendations for launch vibration testing of components for the Mars 2020 rover Sample Caching Subsystem.
			\item Assembled and calibrated a customized benchtop dynamometer to be used for motor/actuator performance testing, documenting assembly process and providing recommendations for improvement to supervising engineer.
			\item Developed procedure and coordinated assembly of epoxy-bonded aluminum honeycomb shock dampers, personally handling and successfully delivering the hardware which was flown on the Mars 2020 Perseverance rover.
		\end{itemize}
	}		
    %
    \datedexperience{NASA Jet Propulsion Laboratory}{Jan 2017 -- Aug 2017}\\[-0.6cm]
	\explanation{Mechanical Engineering Co-op}
	\explanationdetail{%
		\begin{itemize}[label=\textcolor{red}{$\circ$}, leftmargin=*]
			\setlength\itemsep{-0.1em}
			\item Assisted with performance testing of flight-like motors and actuators in extreme temperatures and helped to design and lead special testing to uncover cause of premature gearbox degradation.
			\item Supported R\&D on the proposed Mars Sample Return mission with trade studies and conceptual design development for a robust latching system and alignment/release mechanism.
			\item Developed Python and MATLAB-based data analysis tools to visualize motor performance test data and generate test summaries to be delivered directly to supervising engineers.
		\end{itemize}
	}


% Skills
\section{Technical Skills}
    % %
    % \newcommand{\skillone}{\createskill{CAD/CAM}{SolidWorks \cpshalf NX 11 \cpshalf Slic3r \cpshalf BobCAM \cpshalf ESPRIT}}
    % %
    % \newcommand{\skilltwo}{\createskill{Manufacturing}{Lathing \cpshalf Milling \cpshalf 3D Printing \cpshalf Printed Circuit Boards}}
    % %
    % \newcommand{\skillthree}{\createskill{Programming}{Python \cpshalf C++/Arduino \cpshalf MATLAB \cpshalf JavaScript/HTML/CSS}}
    % %
    % \newcommand{\skillfour}{\createskill{Development}{Git \cpshalf Linux/WSL \cpshalf VS Code}}
    % %
	% \createskills{\skillone, \skilltwo, \skillthree, \skillfour}
	\explanationdetail{\footnotesize \textsc{%
		\begin{tabular}{ r l l }
			\textbf{CAD/CAM Software}				& SolidWorks \cpshalf NX 11 \cpshalf Slic3r \cpshalf BobCAM \cpshalf ESPRIT							\\
			\textbf{Manufacturing Experience}		& CNC Lathing / Milling / Routing \cpshalf 3D Printing (FDM) \cpshalf Printed Circuit Boards				\\
			\textbf{Programming Languages}			& Python \cpshalf C++ / Arduino \cpshalf MATLAB \cpshalf JavaScript \cpshalf  HTML \cpshalf CSS					\\
			\textbf{Development Tools}				& Git \cpshalf Linux / WSL \cpshalf VS Code												
		\end{tabular}
		}
	}

% Awards
\section{Awards and Certifications}
	\explanationdetail{%
		\begin{itemize}[label=\textcolor{red}{$\circ$}, leftmargin=*]
			\setlength\itemsep{-0.1em}
			\item Tau Beta Pi Engineering Honor Society Member, California Lambda Chapter
			\item EIT/FE (California), Certification No. EIT 146122
			\item Eagle Scout, Boy Scouts of America
		\end{itemize}
	}
    % \newcommand{\extraone}{%
    % \lipsum[1][7-8]\par %replace this part with actual text
    % }
    % %
    % \newcommand{\extratwo}{%
    % \lipsum[1][9-10]\par %replace this part with actual text
    % }
    % %
    % \newcommand{\extrathree}{%
    % \lipsum[1][11-12]%replace this part with actual text
    % }
    % %
    % \newcommand{\listofextras}{\extraone, \extratwo, \extrathree}
    % %
    % \createbullets{\listofextras}
%
%Footnote
% \createfootnote
\end{document}
