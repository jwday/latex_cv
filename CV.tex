% !TEX program = xelatex
\documentclass{muratcan_cv}
\usepackage{enumitem}
\usepackage{natbib}
\usepackage{bibentry}
\usepackage{hyperref}
\bibliographystyle{unsrt}

\setname{Josh}{Day}
\setaddress{625 G Street -- Davis, CA}
\setmobile{(916) 765-9396}
\setmail{jwday@ucdavis.edu}
\setposition{Work Student} %ignored for now
\setlinkedinaccount{https://www.linkedin.com/in/josh-day-49b10a58/} %you can play with color of the template (red is also nice..)
\setgithubaccount{https://github.com/jwday} %you can play with color of the template (red is also nice..)
\setthemecolor{Orange} %you can play with color of the template (red is also nice..)
\urlstyle{same}

%----------------------------------------------------------------------------------
%            bibliographyish
%----------------------------------------------------------------------------------
\begin{filecontents}{publications.bib}
	@article{day2020two,
		author={Day, Josh W and Robinson, Stephen K},
		title={{Two-Fault} {Tolerant} {Cold} {Gas} {Propulsion} {System} for {Spacecraft-Inspection} {CubeSat}},
		booktitle={AIAA Scitech 2020 Forum},
		pages={1665},
		year={2020},
		note={DOI: \url{https://doi.org/10.2514/6.2020-1665}}
  	}  
\end{filecontents} 


\begin{document}
%Set variables
%You can add sections, texts, explanations just by copying the style below. Replace the dummy texts "\lipsum[1][x-x]\par" with actual texts.
%Create header
\headerview
\vspace{1em}
%Sections
%
% Summary
% \addblocktext{Summary}{%
% \lipsum[1][1-12]\ %replace this part with actual text
% }
%
% Education
\section{Education}
    \datedexperience{M.S. Mechanical and Aerospace Engineering}{University of California, Davis -- 2020}
	% \explanation{University of California, Davis}
	\explanationdetail{%
		\begin{itemize}[label=\textcolor{red}{$\circ$}, leftmargin=*]
			\item Thesis: \emph{Development of a Two-Fault Tolerant Cold Gas Propulsion System and Air Bearing\newline Testbed for Application to a Spacecraft-Inspection CubeSat}, \href{https://github.com/jwday/Thesis-Pub/raw/master/Josh\%20Day\%2C\%20MS\%20Thesis\%20vFinal.pdf}{available online.}
		\end{itemize}
    }
    \datedexperience{B.S. Mechanical Engineering}{University of California, Davis -- 2013}
    \datedexperience{B.S. Aerospace Science \& Engineering}{}
    % \explanation{University of California, Davis}\\[-0.8cm]


% Publications
\section{Publications}
	\mbox{
		\hspace{1em}
		\begin{minipage}{.85\linewidth}
			\footnotesize
			% \scshape
			\nobibliography{publications}
			\bibentry{day2020two}
		\end{minipage}
	}\\


% Experience
\section{Experience}
	\datedexperience{Human/Robotics/Vehicle Integration \& Performance Lab}{Jun 2016 -- Oct 2020}
	\explanation{Graduate Student Researcher}
	\explanationdetail{%
		\begin{itemize}[label=\textcolor{red}{$\circ$}, leftmargin=*]
			\setlength\itemsep{-0.05em}
			\item Constructed small satellite propulsion system testbed using air bearings and 3D printed components to simulate microgravity conditions, enabling highly repeatable motion measurements at low cost.
			\item Designed and constructed a cold gas propulsion system using 3D printed pressure vessels and commercial off-the-shelf parts to achieve full 3 degrees-of-freedom control of an air bearing test platform.
			\item Implemented Python + OpenCV-based computer vision tracking system to measure air bearing platform motion and utilized Kalman filtering to extract acceleration data for accurate vision-based thrust measurement.
			\item Developed full stack human-machine interface using JavaScript + Arduino  (C++) in Linux environment for real-time wireless control of propulsion system and feedback from sensors.
		\end{itemize}
	}
	%
    \datedexperience{UC Davis Department of Mechanical and Aerospace Engineering}{Sep 2015 -- Jun 2019}
    \explanation{Teaching Assistant (Multiple Courses)}
	\explanationdetail{%
		\begin{itemize}[label=\textcolor{red}{$\circ$}, leftmargin=*]
			\setlength\itemsep{-0.05em}
			% \item As lead TA for Engineering Design \& Communications, assisted with lesson plan development and supervised three bi-weekly lab sessions consisting of 6 teams of 4 students, each with their own individual project needs.
			\item \emph{Engineering Design \& Communications} -- As lead TA, assisted lesson plan development, gave weekly lectures introducing students to sensors and microcontrollers, and provided critical feedback on public speaking and technical communication.
			% \item Applied experience with programming, sensors, and actuators to introduce students to data collection with microcontrollers and guide team projects to address problems identified on the UC Davis Student Farm.
			% \item Provided effective feedback to students engaged in public speaking and presentations, citing notable improvement in technical communication and public speaking skills from beginning to end of quarter.
			\item \emph{Measurement Systems} -- Supervised weekly lab sessions introducing students to electro-mechanical sensors such as op-amps and strain gauges to measure bending stress and accelerometers to identify vibrational modes.
			\item \emph{Manufacturing Processes} -- Instructed students on safe operation of manual fabrication equipment and proper use of geometric dimensioning and tolerancing, leading each student in the construction of their own personalized gyroscope.
		\end{itemize}
	}
    %
    \datedexperience{NASA Jet Propulsion Laboratory}{Jun 2018 -- Aug 2018}
	\explanation{Mechanical Engineering Intern}
	\explanationdetail{%
		\begin{itemize}[label=\textcolor{red}{$\circ$}, leftmargin=*]
			\setlength\itemsep{-0.05em}
			\item Designed components to emulate mass properties of flight hardware and provided sensor recommendations for launch vibration testing of components for the Mars 2020 rover Sample Caching Subsystem.
			\item Assembled and calibrated a customized benchtop dynamometer to be used for motor/actuator performance testing, documenting assembly process and providing recommendations for improvement to supervising engineer.
			\item Developed procedure and coordinated assembly of epoxy-bonded aluminum honeycomb shock dampers, personally handling and successfully delivering the hardware which was flown on the Mars 2020 Perseverance rover.
		\end{itemize}
	}		
    %
    \datedexperience{NASA Jet Propulsion Laboratory}{Jan 2017 -- Aug 2017}
	\explanation{Mechanical Engineering Co-op}
	\explanationdetail{%
		\begin{itemize}[label=\textcolor{red}{$\circ$}, leftmargin=*]
			\setlength\itemsep{-0.05em}
			\item Assisted with performance testing of flight-like motors and actuators in extreme temperatures and helped to design and lead special testing to uncover cause of premature gearbox degradation.
			\item Supported R\&D on the proposed Mars Sample Return mission with trade studies and conceptual design development for a robust latching system and alignment/release mechanism.
			\item Developed Python and MATLAB-based data analysis tools to visualize motor performance test data and generate test summaries to be delivered directly to supervising engineers.
		\end{itemize}
	}


% Skills
\section{Technical Skills}
    % %
    % \newcommand{\skillone}{\createskill{CAD/CAM}{SolidWorks \cpshalf NX 11 \cpshalf Slic3r \cpshalf BobCAM \cpshalf ESPRIT}}
    % %
    % \newcommand{\skilltwo}{\createskill{Manufacturing}{Lathing \cpshalf Milling \cpshalf 3D Printing \cpshalf Printed Circuit Boards}}
    % %
    % \newcommand{\skillthree}{\createskill{Programming}{Python \cpshalf C++/Arduino \cpshalf MATLAB \cpshalf JavaScript/HTML/CSS}}
    % %
    % \newcommand{\skillfour}{\createskill{Development}{Git \cpshalf Linux/WSL \cpshalf VS Code}}
    % %
	% \createskills{\skillone, \skilltwo, \skillthree, \skillfour}
	\explanationdetail{\footnotesize \textsc{%
		\begin{tabular}{ r l l }
			\textbf{CAD/CAM Software}				& SolidWorks \cpshalf NX 11 \cpshalf Slic3r \cpshalf BobCAM \cpshalf ESPRIT							\\
			\textbf{Manufacturing Experience}		& CNC Lathing / Milling / Routing \cpshalf 3D Printing (FDM) \cpshalf Printed Circuit Boards		\\
			\textbf{Programming Languages}			& Python \cpshalf C++ (Arduino) \cpshalf MATLAB \cpshalf											\\
			\textbf{Development Tools}				& Git \cpshalf Linux / WSL \cpshalf VS Code												
		\end{tabular}
		}
	}

% Awards
% \section{Awards and Certifications}
% 	\explanationdetail{%
% 		\begin{itemize}[label=\textcolor{red}{$\circ$}, leftmargin=*]
% 			\setlength\itemsep{-0.1em}
% 			\item Tau Beta Pi Engineering Honor Society Member, California Lambda Chapter
% 			\item EIT/FE (California), Certification No. EIT 146122
% 			\item Eagle Scout, Boy Scouts of America
% 		\end{itemize}
	% }
    % \newcommand{\extraone}{%
    % \lipsum[1][7-8]\par %replace this part with actual text
    % }
    % %
    % \newcommand{\extratwo}{%
    % \lipsum[1][9-10]\par %replace this part with actual text
    % }
    % %
    % \newcommand{\extrathree}{%
    % \lipsum[1][11-12]%replace this part with actual text
    % }
    % %
    % \newcommand{\listofextras}{\extraone, \extratwo, \extrathree}
    % %
	% \createbullets{\listofextras}
	
\pagebreak

\section{Key Projects}
	\datedexperience{\href{https://github.com/jwday/ABTestbed}{\color{black}{Air Bearing-based Propulsion System Testbed \faGithubSquare}}}{}
	\explanationdetail{%
		An air bearing-based platform and table was developed to provide a free-floating low friction environment for testing a prototype cold gas propulsion system.
		Pneumatic components were acquired off-the-shelf and structural components 3D printed to reduce cost and enable facilitate dissemination of open-sourced part files.
		The air bearing surface was constructed by placing a sheet of regular window pane glass atop an optics table and shimmed as needed to achieve a functionally level surface.
		% The platform was found to operate marginally at 30 psig with improved functionality up to 50 psig, after which no performance improvements were observed.
		It was found that debris and inconsistencies in the table surface were the main contributors to degraded performance, and new facilities are being constructed to house the table and mitigate surface debris.
		% The system was required to operate independent of any external power or gas supplies and maintain a small footprint to maximize usable table space.
		% The design of the platform was driven by performance and cost constraints of the air bearings
		% Three air bearings were mounted to threaded ball studs in a triangular configuration and held in place by a 3D-printed structure.
		% This same structure supported 2x 9-oz. CO$_{2}$ bottles, a regulator, an air filter, and the propulsion system.
		% The air bearing table was constructed by placing a 0.125''-thick pane of glass atop a 3' x 5' optics bench and shimmed with 0.1-mm Kapton tape to minimize surface deviations.
	}
	%
	\datedexperience{\href{https://github.com/jwday/remoteProp}{\color{black}{Arduino and Raspberry Pi-based Wireless Propulsion Controller \faGithubSquare}}}{}
	\explanationdetail{%
		A propulsion system consisting of twelve solenoid valves and two pressure transducers was controlled using an Arduino-like NodeMCU microcontroller and additional components laid out on a custom-designed printed circuit board (PCB).
		The NodeMCU was chosen as a low-cost platform which could wirelessly communicate with a message broker using a simple MQTT framework.
		The Raspberry Pi simultaneously hosted the MQTT message broker and an HTTP server to allow clients to send commands to the propulsion system via a web interface.
		The same framework was used to support data collection during thrust testing by wirelessly relaying strain gauge and pressure measurements to the broker which then stored the data in a database for post-processing.
	}
	%
	\datedexperience{\href{https://github.com/jwday/ComputerVision}{\color{black}{Computer Vision System and Kalman Filter Data Post-processor \faGithubSquare}}}{}
	\explanationdetail{%
		Position, velocity, and acceleration of a free-floating air bearing platform was measured using a Python-based computer vision algorithm on recorded video.
		The algorithm detects a fiducial marker placed on the platform and returns a timeseries dataset of relative pose for each frame in the video.
		The dataset is passed to a Kalman filter which combines the data with a kinematic model to estimate the velocity and acceleration of the platform over time.
		These tools allow a user to detect accelerations due to thrust and friction, and quantitatively compare the effects of different combinations of thruster failures.
	}
	%
	\datedexperience{3D Printed Propellant Pressure Vessel}{}
	\explanationdetail{%
		Propellant storage volumes ('plenums') were constructed to store CO2 propellant at 100 psig using an off-the-shelf benchtop 3D printer.
		A process was developed to finish and seal the 3D printed components with two-part epoxy and securely bond a threaded fitting to allow the plenums to be mated directly to the propulsion control valves.
		Aided by CAD-based stress analysis, several iterations were produced with varying connectors, geometry, and print methods in order to produce a final version which could be reliably printed in a short amount of time.
		Each plenum underwent proof testing to double its expected pressure (200 psig) using water to minimize explosive energy in event of failure.
	}
	\datedexperience{\href{https://github.com/jwday/nozzleDesign}{\color{black}{Gas Dynamics Simulation for Single Plenum Discharge \faGithubSquare}}}{}
	\explanationdetail{%
		A Python-based gas dynamics simulation was developed to predict the performance characteristics of the plenum-based propulsion system by modeling the change in propellant thermodynamic properties during a discharge process.
		The change in gas properties with each time step was modeled assuming an isentropic process for an ideal gas, and the corresponding thruster performance was calculated assuming an isentropic process through the nozzle.
		Further development replaced the isentropic model inside the plenum with interpolated thermodynamic data for real gasses to more accurately predict the temperature change during a discharge.
		Simulation results were experimentally validated by measuring pressure and thrust during a discharge, which were found to be in good agreement after the completion of an initial start-up phase whereby the mass flow rate and thrust increased to a maximum before decaying.
		% The modeled process was found to be in the same family as that of a rocket nozzle during its shut-off phase, with the exception of a brief period in the beginning where mass flow rate and thrust must first rise to a maximum before decaying.
	}
	\datedexperience{\href{https://github.com/jwday/rotating-torus-sympy}{\color{black}{Dynamics Modeling of a Rotating Torus with Time-Varying Mass Distribution \faGithubSquare}}}{}
	\explanationdetail{%
		A Multibody Dynamics class project which used Python to implement Kane's Method of dynamic analysis to model a rotating torus in free space as system mass is shifted to offset the center of gravity.
		The motivation for this project was to explore the use of rotation to simulate gravity in a theoretical space station design and explore how internal disturbances could affect the dynamics of the rotation.
		A simply-rotating body with point masses around the circumference was modeled using the appropriate formulation, but the specific challenge lied in accessing the location of the point masses during the integration routine and changing them mid-step.
		As expected, entirely in-plane motion resulted in simple translation of the center of mass while rotation remained in-plane.
		More interestingly, some out-of-plane motion of the point masses resulted in a visible ``wobble'', necessitating the introduction of counter-balance masses which would be used to preserve system stability.
	}




%Footnote
% \createfootnote
\end{document}
